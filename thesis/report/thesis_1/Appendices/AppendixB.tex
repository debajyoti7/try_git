% Appendix Template

\chapter{Barab\'asi Albert Model} % Main appendix title

\label{AppendixB} % Change X to a consecutive letter; for referencing this appendix elsewhere, use \ref{AppendixX}

\lhead{Appendix X. \emph{B-A Model}} % Change X to a consecutive letter; this is for the header on each page - perhaps a shortened title

This algorithm for generating random scale-free networks was developed in 1999.
This model was developed when researching the structure of the World Wide Web and the initial expectations of the researchers was to find a random network, but instead, they witnessed the existence of Power-Law in the degree distribution of the nodes, and the concept of scale-free networks was born.

This structure explain the presence of a few highly-connected hubs and a larger number of less connected nodes in the networks, and the architecture holds true for a wide range of networked structures, varying from a man-made physical infrastructures like the internet to the biological networks of proteins.

This model also emphasises on the role of hubs, which is very important and have applications in fields ranging from IT security to spread of contagious diseases and more. In words of Albert-L\'aszl\'o Barab\'asi, the hubs make the scale-free networks ``very robust to random failures, but very fragile to targeted attacks". 
