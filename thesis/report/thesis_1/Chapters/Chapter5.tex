% Chapter Template

\chapter{Significance of Results} % Main chapter title

\label{Chapter5} % Change X to a consecutive number; for referencing this chapter elsewhere, use \ref{ChapterX}

\lhead{Chapter 5. \emph{Significance of Results}} % Change X to a consecutive number; this is for the header on each page - perhaps a shortened title

%----------------------------------------------------------------------------------------
%	SECTION 1
%----------------------------------------------------------------------------------------

\section{The Model}
The model shows us a fast and easy way to generate a scale-free network, and the effects of using the constraint of `likemindedness', defined in chapter~\ref{Chapter3} section~\ref{sec:definitions}, between the agents, which reduces the number of connections formed significantly, but the network still follows power law.

Although this thesis is focused on companies, the modularity of the project allows the algorithm~\ref{alg1} to be used for generic simulation of social structures.

%----------------------------------------------------------------------------------------
%	SECTION 2
%----------------------------------------------------------------------------------------

\section{The Simulation}

The initial results for diffusion in a scale-free network present a picture where it is clear that the spread, whether it is increase or decrease, tends to be exponential. 

The simulation results presented in chapter~\ref{Chapter4} section~\ref{sec:cost_gain} give a varied picture. They are divided in two groups.

%-----------------------------------
%	SUBSECTION 1
%-----------------------------------
\subsection{Base Group}

This group focuses on analysis while cost is same for both companies, and helps the observer to understand the basic nature of the companies. 

The base group analyses shows that company2 seems to emerge victor for overall simulation, however, the nature is unstable for shorter runs, i.e., during the initial time steps of the simulations.


%-----------------------------------
%	SUBSECTION 2
%-----------------------------------

\subsection{Difference Group}

This group focuses on analysis while cost is different for both companies, and allows for analysis of the different policies.

The author focused on two types of policies described in chapter~\ref{Chapter2} section~\ref{sec:policies}. Specifically, the chosen policies were :

\begin{enumerate}
\item[Company1 :] Symbolic involvement. User input is requested but ignored.
\item[Company2 :] Involvement by doing. A user as design team member or as the official liaison with the information system’s development group.
\end{enumerate}


Company1 maintained a minimal involvement of customers in decision making, and made very little effort. This made sure that the company did not take a very high cost. 

Company2 had a more open policy where customers were highly involved, and the company made a lot of efforts to reach out. This resulted in formation of long-lasting relations which led to profit in the long run.

The analyses show that company1, with its minimal inclusion policy, is more effective for short-term projects, when it makes less efforts towards the customers. The policy followed by Company2 is less profitable during this phase, but it starts to pay-off afterwards and becomes more profitable than Company1.

The exact shift point for this change is case dependant, and it varies with the value of gain, and costs. But this model could be further developed to prepare prediction system that helps companies to switch between policies in order to maximize their cost-benefit ratio.
