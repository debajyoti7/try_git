% Chapter 1

\chapter{Introduction} % Main chapter title

\label{Chapter1} % For referencing the chapter elsewhere, use \ref{Chapter1} 

\lhead{Chapter 1. \emph{Introduction to the thesis}} % This is for the header on each page - perhaps a shortened title

%----------------------------------------------------------------------------------------

\section{Networks}

\subsection{Computer Networks}
Computer Networks are networks of interconnected devices that can exchange and share information and resources. These connections may be categorized depending on the form of connections being formed, and rules governing communications within the network.

\subsection{Social Networks}
Social Networks refer to arrangement of individual agents, most commonly humans, such that these arrangements are dependent on the actions of the agents and also influence their future actions. This is also referred to as a `social structure'. 

With the personalization of technology, computer networks are shaping themselves to get even closer to the social networks in terms of structure, shape, size, and degree distribution, for example. A comprehensive review of the same was done by Newman~\cite{Newman03thestructure} . 

This is particularly important and interesting with the emergence of personalized wearable devices. This could lead to a 1:1 man to machine ratio, which would mean, when allowed, the personal devices would network exactly as their users.

\subsection{Scale-Free Networks}
A scale-free network is a network with the property that the fraction P(k) of nodes with `k' connections  is determined as: 

\begin{equation}
P(k) \sim k^{\-\gamma}
\label{eqn:power law}
\end{equation}

In general, the degree distribution of the network follows a power law. 



\section{State of the Art}
Everything networks. Studies have shown networks to evolve over time to optimize functionalities, and increase longevity. The most coveted type of network, perhaps, is the kind formed naturally among varied entities, i.e., free scaling networks ~\cite{1999Sci...286..509B, 2002RvMP...74...47A}.

The first model for free scaling networks was proposed by Barab{\'a}si–Albert~\cite{1999Sci...286..509B}, hereinafter referred to as B-A model, and the works of Erd\H{o}s \& R\'{e}nyi~\cite{Erdos60onthe}, Watts and Strogatz~\cite{1251797}, Watts, Strogatz and Newman~\cite{PhysRevE.64.026118},  Saramäki  and Kaski~\cite{Saramaki200480} , Yang and Jure~\cite{Yang_modelinginformation} , and  Rycroft~\cite{Rycroft2007565} also helped greatly in laying the groundwork.

It is prominent that existing social networks tend to scale freely and it is safe to assume that personalized devices, if and when allowed to network, will follow a similar trace.

\subsection{State of the art in networked devices}

IoT~\cite{Internet-of-things} - As the almighty Internet steps out of traditional computers to directly link everyday physical objects, overcoming the spatio-temporal boundations, it upgrades a part of our lives to the so called cloud.
This makes our lives easier, but also raises confusion as the complexity of processes increase manifold. 

CPS and M2M~\cite{6601317} - IoT is but a small part of a larger picture, CPS (Cyber-Physical Systems), where computational devices of all size and shape interact with each other and with everyday objects to perform complex tasks, ranging from temperature control in a modern house, to automatic detection of spread of a potentially fatal epidemic.
M2M makes it possible for sensors spread over a large area to share the load of detecting varied signals, while some entirely different processing entity looks into that raw data and extracts meaningful information from it, and then a strong/special link actuates some physical entity to act based on the  information just gathered.However, large scale CPS faces a major challenge due to Heterogeneous nature of network elements.

\subsection{Examples of State of the Art networked devices }
\begin{enumerate}

\item GremlinMusic ~\cite{gremlin}

Gremlin showed a concept of interconnecting embedded devices in a whole new light. It not only allowed users to carry their music along (like every music player), but it allowed friends to connect their Gremlins and legally share music with each other. It allowed for an optimization of storage, bandwidth (in p2p form), and monetary resources for the users.  An analogy for the same could be to take an iPod and put facebook and a free Spotify premium on it.

\item p-Cell technology ~\cite{pCell}

The new technology by Artemis seems promising, and could have crucial impact on the state of networked devices over time.


\item Swarm robots 

Swarm robots are a good example of how even heterogeneous entities can work together, like the coordination between the Eye-Feet-Hand bots~\cite{swarm..robots} to achieve their goal.

\end{enumerate}



%----------------------------------------------------------------------------------------

\section{Motivation}

Since networks are everywhere, it's important we understand their nature and working so as to exploit and utilize them to our benefit.  
Going back as far as the 18th century, the ``Seven Bridges of Königsberg"~\cite{konigsberg} might be the most famous networking problem. Ranging from the Travelling-Salesman~\cite{traveling} to Graph-Coloring algorithms~\cite{graph}, insight into the working of networks have helped greatly in optimizing several issues.

\subsection{Why Scale-Free Networks}

Scale-Free networks are the most prevalent in nature. Hence, it is paramount to model networking of personal devices used by humans on the same topology. 
This model aims to gain insight into the workings of such networks, so as to optimize the network elements and make use of their full potential. For eg., by finding the optimal positioning for a self-assembling network of satellites, military installations can provide improved security at a lesser cost. Or multiple groups of Swarm networks, like groups of eye-hand-feet bots~\cite{swarm..robots}, can work together as a collective with an increased proficiency.  Most importantly, due to the autonomous nature of most network elements, these networks can be formed anywhere, water, air, or land, and can accomplish a multitude of tasks.

\subsection{Human to Human}
Human beings are innately complex, yet strive for simplicity~\cite{h2h}. The challenge today as humans is to find, understand and explain the complex in its most simplistic form. This is important not only for businesses, but for overall development of human society to a point where communications may have the desired effect to a greater extent.


\subsection{Future Sight}

Imagine if smart phones were a bit smarter, and they synchronized the user's calender to include events from his/her best friend's public calender automatically, or if it alerted the user that the person they are supposed to meet has now entered the building.

Or imagine a world where every child has a personal monitoring device, in form of a wearable device, and as the children go to the playground in the evening, these devices are with them. Now, not only do they serve the purpose of monitoring the child, but these devices communicate with each other and form a bond, in a most likely manner that the children form their bonds, so while the kid is busy playing, his/her device could find out about the studies they missed at school.

Networked devices could make life easier and yet more manageable.


%----------------------------------------------------------------------------------------

\section{Aim}
This thesis aims to analyse diffusion of innovation in the society. 
The author simulates a real-world scenario where companies try to influence customers by spreading an idea or publicizing their products. As an effect, the customer agents, hereinafter referred to as `agents' only, tend to lean towards the company if they are successfully , and enough, influenced.
However, this campaigning propaganda incurs some costs to the companies. The author tries to find a balance between gain and cost for the company, in order to formulate a strategy which can be effectively used by the company to optimally attract customers. 

Different companies have different policies towards customer involvement and making an effort towards further attracting the customer. The objective here is to analyze and evaluate two of such policies, and to determine which policy is optimal for the given circumstances.
