% Chapter 1

\chapter{Introduction} % Main chapter title

\label{Chapter1} % For referencing the chapter elsewhere, use \ref{Chapter1} 

\lhead{Chapter 1. \emph{Introduction to the thesis}} % This is for the header on each page - perhaps a shortened title

%----------------------------------------------------------------------------------------

\section{State of the Art}
Everything networks. Studies have shown networks to evolves over time to optimize functionalities, and increase longevity. The most coveted type of network, perhaps, is the kind formed naturally among varied entities, i.e., free scaling networks.

The first model for free scaling networks was proposed by BA , and the works of [REFERENCES] also helped in laying the groundwork.

\subsection{State of the art in networked devices}

IoT - As the almighty Internet steps out of traditional computers to directly link everyday physical objects, overcoming the spatio-temporal boundations, it upgrades a part of our lives to the so called cloud.
This makes our lives easier , but also raises confusion as the complexity of processes increase manifold. 

CPS and M2M - IoT is but a small part of a larger picture, CPS (Cyber-Physical Systems), where computational devices of all size and shape interact with each other and with everyday objects to perform complex tasks, ranging from temperature control in a modern house, to automatic detection of spread of a potentially fatal epidemic.
M2M makes it possible for sensors spread over a large area to share the load of detecting varied signals, while some entirely different processing entity looks into that raw data and extracts meaningful information from it, and then a strong/special link actuates some physical entity to act based on the  information just gathered.However, large scale CPS faces a major challenge due to Heterogeneous nature of network elements

\subsection{Examples of State of the Art networked devices }
1. GremlinMusic [REFERENCES] : Gremlin showed a concept of interconnecting embedded devices in a whole new light. It not only allowed users to carry their music along (like every music player), but it allowed frends to connect their Gremlins and legally share music with each other. An optimization of storage, bandwidth (in p2p form), and monetary resources for the users.  Analogy : take an iPod and put facebook and a free Spotify premium on it.

2. p-Cell technology [REFERENCES] : the new technology by Artemis seems promising, and could have crucial impact on the state of networked devices over time.

3. Swarm robots [REFERENCES] : Swarm robots are a good example of how even heterogeneous entities can work together. Like the coordination between these Eye-Feet-Hand bots to achieve their goal



%----------------------------------------------------------------------------------------

\section{Motivation}

Since networks are everywhere, it's important we understand their nature and working so as to exploit and utilize them to our benefit.  
Going back as far as the 18th century, the "Seven Bridges of Königsberg" might be the most famous networking problem. Ranging from the Travelling-Salesman to GRaph-Coloring algorithms, insight into the working of networks have helped greatly in optimizing several issues.

\subsection{Why Scale-Free Networks}

Scale-Free networks are the most prevalent in nature. Hence, it is paramount to model networking of personal devices used by humans on the same topology. 
This model aims at gaining insight into the workings of such networks, so as to optimize the network elements and make use of their full potential. For eg., by finding the optimal positioning for self-assembling network of satellites, military installations can provide improved security at a lesser cost. Or multiple groups of Swarm networks, like groups eye-hand-feet bots, can work together as a collective with an increased proficiency.  Most importantly, due to autonomous nature of most network elements, these networks can be formed anywhere, water, air, or land, and can accomplish a multitude of tasks.

\subsection{Future Sight}

Imagine if smart phones were a bit smarter, and it synchronized the user's calendar to include events from his/her best friend's public calender automatically, or if it alerted the user that the person they are supposed to meet has now entered the building.
Or imagine a world where every child has a personal robot, in form of a pet, or a caretaker, and as the children go to the playground in the evening, these bots go with them, now, not only they serve the purpose of monitoring the child, these bots communicate with each other and form a bond, in a most likely manner that the children form bonds, so while your son is busy playing, his robot could find out about the studies he missed at school because he had a stomachache that morning.
Networked devices could make life easier and yet more manageable.
	
%----------------------------------------------------------------------------------------
\section{Aim}
This thesis aims to find optimal parameters for diffusion of innovation in the society. 
The author simulates a real-world scenario where companies try to influence it's customers by spreading an idea or publicizing their products, and as an effect the agents tend to lean towards the company if they are successfully , and enough, influenced.
However, this campaigning propaganda incurs some costs to the company. The author tries to find a balance between the gain and cost for the company, in order to formulate a strategy which can be effectively used by the company to optimally attract customers. 

\begin{flushright}
Guide written by ---\\
Sunil Patel: \href{http://www.sunilpatel.co.uk}{www.sunilpatel.co.uk}
\end{flushright}
