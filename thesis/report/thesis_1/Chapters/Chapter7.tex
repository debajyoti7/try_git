% Chapter Template

\chapter{Summary and Conclusion} % Main chapter title

\label{Chapter7} % Change X to a consecutive number; for referencing this chapter elsewhere, use \ref{ChapterX}

\lhead{Chapter 7. \emph{Summary and Conclusion}} % Change X to a consecutive number; this is for the header on each page - perhaps a shortened title

%\section{The Need}

Technology is ever evolving and the nature of this evolution has an inseparable bond with the users. The need for understanding diffusion in a social structure is very significant in the present day world, where technology is all around and getting ever closer to humans in every aspect. This understanding would allow for the latest technological evolutions to be understood and exploited to their full potential.

%\section{The Model}
In this thesis, the author discusses the impact of various policies followed by companies to involve customers in their decision-making strategies. He discusses significance of the chosen social structure, a scale-free network, and presents methods to create a scale-free network fit for his purpose by following a slightly modified form~\ref{alg1} of the B-A Algorithm. He continues his work by simulating an environment where the agents participate in diffusion of various signals.

%\section{The Simulation}
The author simulates two companies competing for the attention of the agents, by trying to spread their propaganda. The companies take initiative by making an effort to move towards the agents, while the agents show their inclination by changing their color to get closer to the color representing the company. The effort made by the companies requires some expenditure, in terms of cost to the company and the inclination of the agents result in gain for the specific company. He uses the data gathered from the simulations to analyse the policies followed by the companies while attracting the customers, and presents his findings which suggest a basic model for maximizing gain to cost ratio for the companies.

%\section{Results}
The author concludes his work by showing that depending on per unit gain and cost, it is more effective to form long-term relations with the customers for stable or continuous assignments while periodically changing assignments, for e.g.. seasonal businesses, might reap more profit by not involving the customers to a high degree. 

While the presented analyses are focused on marketing campaigns, the author emphasises on the importance of the work by discussing its relation to the real-world scenario and similarity to real-life. He also suggests that a similar approach could be used in wearable devices to let devices connect and map the social structure of their users, and this model would provide analytical ability over such a network. 
