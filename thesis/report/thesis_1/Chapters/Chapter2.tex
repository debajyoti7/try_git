% Chapter Template

\chapter{Approach and Tools} % Main chapter title

\label{Chapter2} % Change X to a consecutive number; for referencing this chapter elsewhere, use \ref{ChapterX}

\lhead{Chapter 2. \emph{Approach and Tools}} % Change X to a consecutive number; this is for the header on each page - perhaps a shortened title

%----------------------------------------------------------------------------------------
%	SECTION 1
%----------------------------------------------------------------------------------------

\section{Approach}


%-----------------------------------
%	SUBSECTION 1
%-----------------------------------
\subsection{Network Model}
It was an obvious choice to select scale-free topologies for modeling the network structure.  But B-A model also proposes variants of the scale free structure depending on it's characteristics. These characteristics are : 

\begin{enumerate}
\item Continuous Growth 

This states that the network grows continuously. i.e., at all times, new nodes are being attached to the the network.


\item Preferential Atachment

This states the rule any node should follow while making connections. It states that the most connected nodes are most likely candidates to form a connection with.


\end{enumerate} 

Now, a scale free network may exhibit either or both of above characteristics. However, the author decided to include both in his model as to make it as close to real life as possible. 

%-----------------------------------
%	SUBSECTION 2
%-----------------------------------
\subsection{Orientation and View of the simulation environment}

Initially, an object oriented view was used to give better control over the network agents. But this led to higher time complexity, and the author decide to switch to a connection-view model, where more focus was given to the connections being formed and everything was managed from that view. This resulted in significant decrease in time complexity.
Also, focusing on the connections was easier as the whole network could be minimally represented by using the edge list.


%-----------------------------------
%	SUBSECTION 3
%-----------------------------------

\subsection{Simulation Approach}

The attribute \emph{color} was chosen for the simulation as it is easy to understand and visualise, and let's us present the effect in a 2-dimensional model where clustering is not directly dependent on the axial values. 
As a limiting case, the author decided to simulate only two companies. This limits the scope as it does not give much insight into cases where two companies might collaborate for mutual benefit, like to win against a third rival company.


%-----------------------------------
%	SECTION 2
%-----------------------------------

\section{Tools}
As a requirement from the University, Matlab was chosen as the programming language, and no special toolboxes were used.
The curve fitting app was used occasionally to check the output of the simulation.

The custom formulas used by the author were sometimes first tested as a prototype in Python with NetworkX and Mathematica for mathematical validation. However, any of those implementations do not directly contribute to the result.
