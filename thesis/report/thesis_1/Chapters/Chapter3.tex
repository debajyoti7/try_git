	% Chapter Template

\chapter{Methodology} % Main chapter title

\label{Chapter3} % Change X to a consecutive number; for referencing this chapter elsewhere, use \ref{ChapterX}

\lhead{Chapter 3. \emph{Methodology}} % Change X to a consecutive number; this is for the header on each page - perhaps a shortened title

On an abstract level, the model consists of an environment comprising of numerous agents, and a few companies. What we focus on are the specific strategies used by the companies to attract the agents, and try to measure it's efficiency in terms of the related costs and benefits. Based on these measurements, a company could determine the optimal strategy to maximize it's efficiency for the targeted section or sub-subsection of agents.

%----------------------------------------------------------------------------------------
%	SECTION 1
%----------------------------------------------------------------------------------------

\section{The Companies}

The companies only exist in a superficial form in the model. In this model, each company plays the role by providing a central point for the agents to cluster about, and the movement of this point itself denotes effort made by the company, which in turn incurs a cost to the company.

The author decided to keep this model limited to two companies competing for the same spot, i.e., any agent cannot totally belong to both companies at the same time. 


%-----------------------------------
%	SECTION 2
%-----------------------------------
\section{The Agents}

The agents in this model are simpler constructs, each  having a "color" attribute, and this attribute changes as the agents tend to believe in either of the companies.

Structurally, each agent has following attributes :

\begin{enumerate}
\item ID
\item x-position coordinate
\item y-position coordinate
\item Color value
\item Influence value
\end{enumerate}

%-----------------------------------
%	SECTION 3
%-----------------------------------

\section{connections}
The connections between the agents form in such a manner that the model satisfies the requirements of free-scaling. 
Following in the footsteps of the giants, the implementation is based on the B-A algortithm [REF] .

%----------------------------------------------------------------------------------------
%	SECTION 4
%----------------------------------------------------------------------------------------

\section{Algorithms}

\begin{algorithm}
\caption{Create Scale-Free Network}
\label{ag1}
\begin{algorithmic}
\STATE N : Input, total number of agents
\STATE D : Input, average number of connections
\IF {$D < N+1$}
	\STATE Form a \emph{fully connected} network with D+1 agents
\ENDIF 
\STATE implement $CG$
\STATE $i \gets D+1$
\WHILE{$i < N$} 
	\STATE $Agent_i \gets newAgent$
	\STATE assign $likemindedness$ [FORM-REF]
	\STATE implement $PA$
	\STATE $k \gets 0$
	\WHILE{$k \leq \frac{D}{2}$}
		\STATE calculate $probability_{basic}$ [FORM-REF]
		\STATE compute $weight$ [FORM-REF]
		\STATE $probability_{attachment} \gets  probability_{basic} + weight$ 
		\STATE $Agent_i$ forms $connection_k$ , based on $probability_{attachment}$
	\ENDWHILE	
\ENDWHILE
\end{algorithmic}
\end{algorithm}

\clearpage

\begin{algorithm}
\caption{Simulation}
\label{alg2}
\begin{algorithmic}
\STATE assign $Company$ centres
\STATE calculate $influence$ [FORM-REF]
\STATE assign \emph{color} to all agents
\STATE seed $20\% $ agents to $color1$
\STATE seed $20\%$ agents to $color50$
\STATE seed $40\%$ agents to $color25$
\STATE seed $2 \emph{most influential}$ agents to different companies
\STATE $J$ : number of iterations
\STATE $i \gets 0$
\WHILE{$i \leq J$}
	\STATE select $edge_i$
	\IF{ $FromNode_{color}$ = $\emph{color1}$ OR $\emph{color50}$ }
		\STATE select $threshold$
		\IF{ $ FromNode_{influence} \geq threshold $ }
			\STATE $ToNode_{color} \gets ToNode_{color} + \emph{shift}$
			\STATE $ToNode_{x-coordinate} \gets ToNode_{x-coordinate} + \emph{shift}$
			\STATE $ToNode_{y-coordinate} \gets ToNode_{y-coordinate} + \emph{shift}$
			\STATE $Company_{x-coordinate} \gets Company_{x-coordinate} + \emph{shift}$
			\STATE $Company_{y-coordinate} \gets Company_{y-coordinate} + \emph{shift}$
		\ENDIF
	\ENDIF
\ENDWHILE
\end{algorithmic}
\end{algorithm}

\clearpage

\section{Definitions}
\begin{enumerate}
\item \emph{fully connected}
\item \emph{likemindedness}
\item \emph{weight}
\item \emph{Company centre}
\item \emph{influence}
\item \emph{shift}
\end{enumerate}